\documentclass[12pt]{article}

% --- Página y tipografía ---
\usepackage[letterpaper,margin=2.5cm]{geometry}
\usepackage[T1]{fontenc}
\usepackage[utf8]{inputenc} % si compilas con pdfLaTeX
\usepackage{lmodern}
\usepackage{microtype}

% --- Imágenes y color ---
\usepackage{graphicx}
\usepackage{xcolor}

% --- Control fino de espacios ---
\usepackage{setspace}
\setlength{\parindent}{0pt}

\begin{document}
\thispagestyle{empty}

% ===== Encabezado con logos + texto =====
\begin{minipage}[c]{0.18\textwidth}
    \centering
    % Cambia por tu logo izquierdo
    \includegraphics[width=0.95\linewidth]{img/logo_usac.jpeg}
\end{minipage}
\hfill
\begin{minipage}[c]{0.60\textwidth}
    \small
    Universidad de San Carlos de Guatemala\\
    Escuela de Ciencias Físicas y Matemáticas\\
    Nombre estudiante: Paula Isabel Aguilar Polo\\
    Carnet: 202604936\\
    Programación 1\\
\end{minipage}
\hfill
\begin{minipage}[c]{0.18\textwidth}
    \centering
    % Cambia por tu logo derecho
    \includegraphics[width=1.4\linewidth]{img/logo_ecfm.jpg}
\end{minipage}

\vspace{0.5cm}

% Línea horizontal superior (gruesa)
\noindent\rule{\textwidth}{1.2pt}

\vspace{0.2cm}

% ===== Título =====
\begin{center}
    {\Large\scshape Titulo}\\[0.3em]
\end{center}

\vspace{0.1cm}

% Fecha
\begin{center}
    \small\scshape 29 de enero de 2026
\end{center}

\vspace{0.2cm}

% Línea horizontal inferior (gruesa)
\noindent\rule{\textwidth}{1.2pt}

\vspace{0.6cm}

% ===== Caja de resumen =====
\noindent
\colorbox{gray!35}{%
    \parbox{\textwidth}{%
        \vspace{0.6em}
        \textbf{Resumen}\\[0.3em]
        \small
El siguiente documento trata sobre la importancia de la computación
en mi carrera y en mis planes a futuro.

        \vspace{0.8em}
    }%
}


\section{Objetivos}
Estudiar en el área de física.

\section{INTRODUCCIÓN}
Mi objetivo es estudiar Licenciatura en Física en la Universidad de San 
Carlos.
A pesar de que aun no he planteado qué haría tras terminar la 
 licenciatura, tengo 
 algunas ideas, y en todas me serviría saber sobre computación. Por 
 ejemplo, una de mis ideas principales sería sacar mi maestría en 
 física cuántica. Otra idea que me agrada mucho sería estudiar astrofísica.
 Ambas me interesan y considero que si logro graduarme elegiré alguna de 
las dos para mi maestría, pero aun está por verse, pues puede darse 
el caso
de que algo me interese más.
\section{Desarrollo}
Las razones por las cuales considero que la computación será importante
 en mi área de estudios y de trabajo es porque existen muchos simuladores
  de cuestiones físicas o programas que ayudan a hacer mediciones y 
  análisis. En el caso de la física cuántica, el estudio de partículas 
  requiere de mucha computación avanzada, por lo cual estudiar cursos de
  programación me ayuda a construir una base sobre la cual más adelante
  podré construir estudios aun más sólidos y complejos. Igualmente, en el 
  caso de la astofísica se requiere un alto nivel de computación y 
  programación. Casi con total seguridad se puede afirmar que cualquier
  maestría de esas dos requerirá buenos conocimientos en computación, y
  más adelante lo necesitaré en mi ambiente laboral, ya sea que trabaje 
  en un área de investigación o en áreas más aplicada.
\section{Conclusión}
En conclusión, cualquier área por la que me decida involucra mucha
computación, y empezar desde ahora a construir una base sólida de 
conocimientos me ayudará en el futuro, pues por medio de programas 
se puede hacer una cantidad de cosas impresionante, y se siguen viendo
innovaciones en este área.

\end{document}

\end{document}
