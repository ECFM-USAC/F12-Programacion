\documentclass[12pt]{article}

% --- Página y tipografía ---
\usepackage[letterpaper,margin=2.5cm]{geometry}
\usepackage[T1]{fontenc}
\usepackage[utf8]{inputenc} % si compilas con pdfLaTeX
\usepackage{lmodern}
\usepackage{microtype}

% --- Imágenes y color ---
\usepackage{graphicx}
\usepackage{xcolor}

% --- Control fino de espacios ---
\usepackage{setspace}
\setlength{\parindent}{0pt}

\begin{document}
\thispagestyle{empty}

% ===== Encabezado con logos + texto =====
\begin{minipage}[c]{0.18\textwidth}
    \centering
    % Cambia por tu logo izquierdo
    \includegraphics[width=0.95\linewidth]{img/logo_usac.jpeg}
\end{minipage}
\hfill
\begin{minipage}[c]{0.60\textwidth}
    \small
    Universidad de San Carlos de Guatemala\\
    Escuela de Ciencias Físicas y Matemáticas\\
    Nombre estudiante:\\ Jorge Geovanny Gatica Herrera\\
    Carnet: \\ 202601453\\
    Programación 1\\
\end{minipage}
\hfill
\begin{minipage}[c]{0.18\textwidth}
    \centering
    % Cambia por tu logo derecho
    \includegraphics[width=1.4\linewidth]{img/logo_ecfm.jpg}
\end{minipage}

\vspace{0.5cm}

% Línea horizontal superior (gruesa)
\noindent\rule{\textwidth}{1.2pt}

\vspace{0.2cm}

% ===== Título =====
\begin{center}
    {\Large\scshape Titulo}\\[0.3em]
\end{center}
Utilidad de la programación \LaTeX,  en el análisis de datos
\vspace{0.1cm}

% Fecha
\begin{center}
    \small\scshape 2 de febrero de 2026
\end{center}

\vspace{0.2cm}

% Línea horizontal inferior (gruesa)
\noindent\rule{\textwidth}{1.2pt}

\vspace{0.6cm}

% ===== Caja de resumen =====
\noindent
\colorbox{gray!35}{%
    \parbox{\textwidth}{%
        \vspace{0.6em}
        \textbf{Resumen}\\[0.3em]
        \small
\LaTeX, es una herramienta practica para el registro de documentos en base
 a la divulgación o el desarrollo cientifico, en el área donde quisiera especificarme (Energía nuclear) es muy útil para mantener un registro al día
 de los avances que se han observado a lo largo de la experimentación.\\
        \vspace{0.8em}
    }%
}


\section{Objetivos}
\begin{enumerate}
    \item Definir donde el lenguaje de programación \LaTeX, es útil en el área de 
    generación de energá a base de reacciones físicas nucleares.\\
    \item Resolver cualquier incertidubre acerca del conocimiento y las capacidades del lenguaje de
    programación \LaTeX.\\
    \item Practicar el lenguaje de programación \LaTeX, para el desarrollo personal beneficiando
    distintas áreas de estudio.\\
\end{enumerate}

\section{Marco Teórico}
A continuación se redactarán las aplicaciones prácticas de \LaTeX, se aclara que no es un estudio formalmente
riguroso, pero nos da argumentos y raszones validas para utilizar y aprender \LaTeX.\\
Aunque el uso explicito de \LaTeX, no se menciona en ningún articulo o investigación acerca de generadodes
de energía a partir de la fusión nuclear, podemos deducir varios usos para esta rama de la ciencia. 
Los más comúnes entre todos (como puede ser más evidente) son: el redactar articulos cientificos con un 
formato ya establecido sin tener que hacer mayor esfuerzo y la redacción de medidas de seguridad en un formato 
establecido.\\\\
Algunas alternativas para \LaTeX, es Zenodo. La utilidad que la mayoria de personas en el
mundo es Git. Esta utilidad permite a diferentes maquinas poder trabajar en un mismo
código simultaneamente.\\\\
Git, es la herramienta estándar para gestionar versiones de código en la industria general,
\LaTeX, es solamente una herramienta o utilidad más, existen varias alternativas.

\section{Conclusiones}

\begin{enumerate}
\item LaTeX, es una herramienta que facilita la redacción cientifica, distintas ramas
de la física preieren utilizar diversos lenguajes de programación, aunque no se menciona
explicitamente en ningún artículo, LaTeX, es una buena opción para la divulgación
cietíica.\\

\item LaTeX, es un lenguaje de programación, el cual no tiene una interfaz en sí, existen
varios programas que toleran LaTeX, aunque es más fácil operar en sistemas operativos
"Linux", pero queda a discreción del usuario.\\

\item LaTeX, es "intuitivo" dentro de lo que cabe, con práctica se puede lograr un 
avanzado de codificación.\\

\end{enumerate}

\end{document}
