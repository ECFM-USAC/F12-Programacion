\documentclass[12pt]{article}

% --- Página y tipografía ---
\usepackage[letterpaper,margin=2.5cm]{geometry}
\usepackage[T1]{fontenc}
\usepackage[utf8]{inputenc} % si compilas con pdfLaTeX
\usepackage{lmodern}
\usepackage{microtype}

% --- Imágenes y color ---
\usepackage{graphicx}
\usepackage{xcolor}

% --- Control fino de espacios ---
\usepackage{setspace}
\setlength{\parindent}{0pt}

\begin{document}
\thispagestyle{empty}

% ===== Encabezado con logos + texto =====
\begin{minipage}[c]{0.18\textwidth}
    \centering
    % Cambia por tu logo izquierdo
    \includegraphics[width=0.95\linewidth]{logo_usac.jpeg}
\end{minipage}
\hfill
\begin{minipage}[c]{0.60\textwidth}
    \small
    Universidad de San Carlos de Guatemala\\
    Escuela de Ciencias Físicas y Matemáticas\\
    Nombre estudiante: Mia Zoé Gutiérrez Marroquín\\
    Carnet: 202600433 \\
    Programación 1\\
\end{minipage}
\hfill
\begin{minipage}[c]{0.18\textwidth}
    \centering
    % Cambia por tu logo derecho
    \includegraphics[width=1.4\linewidth]{logo_ecfm.jpg}
\end{minipage}

\vspace{0.5cm}

% Línea horizontal superior (gruesa)
\noindent\rule{\textwidth}{1.2pt}

\vspace{0.2cm}

% ===== Título =====
\begin{center}
    {\Large\scshape Tarea 1}\\[0.3em]
\end{center}

\vspace{0.1cm}

% Fecha
\begin{center}
    \small\scshape 06 de febrero de 2026
\end{center}

\vspace{0.2cm}

% Línea horizontal inferior (gruesa)
\noindent\rule{\textwidth}{1.2pt}

\vspace{0.6cm}

% ===== Caja de resumen =====
\noindent
\colorbox{gray!35}{%
    \parbox{\textwidth}{%
        \vspace{0.6em}
        \textbf{Resumen}\\[0.3em]
        \small

        \vspace{0.8em}
    }%
}

\section{Objetivo}
El presente trabajo contiene un ensayo sobre las áreas de interés en la Física y responde a la pregunta: ¿por qué la programación es importante en Física? A lo largo del documento se analizará la relación existente entre la carrera y la programación, destacando su función como una herramienta fundamental en el análisis de fenómenos físicos. Asimismo, se abordará su relevancia en el desarrollo de avances tecnológicos, investigación científica y en la comprensión de áreas específicas como la astrofísica y las ciencias modernas. 
\newpage
\section{Ensayo}
A la hora de elegir una carrera es muy importante identificar sus ventajas y sus desventaja; lamentablemente muchas personas suelen ser muy poco observadoras con este tipo de cosas. Elegir una carrera no es solo una simple decisión sobre lo que se estudiará en los próximos 5 años, sino que es algo de lo que se dependerá en gran parte de la vida, tanto en lo profesional como en lo personal. Es por eso que al momento de escoger mi carrera tuve que analizar muchas cosas para poder llegar a mi conclusión. Finalmente, decidí estudiar la \textbf{Licenciatura en Física} y con ello aprender de todos los cursos que conforman la carrera, incluyendo eso a \emph{Programación 1}. \\ \\
La carrera de Física es bastante amplia ya que contiene muchas áreas de estudio y de investigación. A pesar de su complejidad y de su variedad, mi interés principal es la astrofísica. Esta rama se encarga de aplicar las leyes de la física para estudiar el universo, su composición, estructura y evolución. Estas investigaciones incluyen el desarrollo de estrellas, galaxias, agujeros negros, materia oscura, energía oscura y la formación planetaria. Sin embargo, no se puede estudiar astrofísica sin la tecnología y la programación. Actualmente, la astrofísica moderna depende estrechamente de la programación para analizar grandes volúmenes de datos, realizar simulaciones complejas y controlar telescopios. La programación es indispensable para identificar patrones, validar diferentes modelos teóricos, de manera que facilita el análisis de información que ocurre en escalas de tiempo y espacio imposibles de analizar en un laboratorio. \\ \\
El uso de la programación en la astrofísica también permite transformar la observación del cielo en conocimiento y datos concretos. Por medio de algoritmos y modelos computacionales, los astrofísicos pueden comparar observaciones reales con predicciones teóricas optimizando el uso de los instrumentos astronómicos y sus procesos de análisis. De esta manera, la programación no solo apoya la investigación, sino que amplía las posibilidades del descubrimiento y la comprensión del universo. \\ \\
Es por eso que mi decisión de estudiar Física surge del interés por el universo y estudiarlo de manera amplia. Dentro de la carrera la astrofísica representa para mí esa \emph{"área única"} valiosa por su capacidad de analizar datos a gran escala. Por lo mismo, este tipo de ciencias no podrían ser estudiadas sin la ayuda de la programación, ya que a lo largo del tiempo hemos descubierto que esta nos ayuda en la creación de simuladores y procesamiento de datos masivos. Por lo que \emph{Programación 1} no es solo un curso más del pénsum de la Licenciatura en Física, sino que es una herramienta fundamental para el desarrollo académico y profesional del  \emph{"futuro Licenciado en Física Aplicada"}.


\end{document}
