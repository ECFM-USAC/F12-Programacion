\documentclass[12pt]{article}


\usepackage[letterpaper,margin=2.5cm]{geometry}
\usepackage[T1]{fontenc}
\usepackage{lmodern}
\usepackage{microtype}
\usepackage{float}




\usepackage{graphicx}
\usepackage{xcolor}

\usepackage{setspace}
\setlength{\parindent}{0pt}

\begin{document}
\thispagestyle{empty}


\begin{minipage}[c]{0.18\textwidth}
    \centering
   
    \includegraphics[width=0.95\linewidth]{img/logo_usac.jpeg}
\end{minipage}
\hfill
\begin{minipage}[c]{0.60\textwidth}
    \small
    Universidad de San Carlos de Guatemala\\
    Escuela de Ciencias Físicas y Matemáticas\\
    Nombre estudiante : Cristian Emmanuell Blanco Cruz\\
    Carnet: 202602924 \\
    Programación 1\\
\end{minipage}
\hfill
\begin{minipage}[c]{0.18\textwidth}
    \centering
  
    \includegraphics[width=1.4\linewidth]{img/logo_ecfm.jpg}
\end{minipage}

\vspace{0.5cm}


\noindent\rule{\textwidth}{1.2pt}

\vspace{0.2cm}


\begin{center}
    {\Large\scshape Área de Investigación: Astrofísica}\\[0.3em]
\end{center}
\vspace{0.05cm}
\begin{center}
    \textbf{Cristian Blanco}
    
\end{center}
\vspace{0.05cm}


% Fecha
\begin{center}
    \small\scshape 4 de febrero de 2026
\end{center}

\vspace{0.2cm}


\noindent\rule{\textwidth}{1.2pt}

\vspace{0.6cm}

\noindent
\colorbox{gray!35}{%
    \parbox{\textwidth}{%
        \vspace{0.6em}
        \textbf{Resumen}\\[0.4em]
        \text{La física en la actualidad se apoya de manera directa en herramientas computacionales para el análisis y modelado de procesamiento de datos complejos. En este documento se describe el interés del autor por la astrofísica como área de investigación, así como la importancia que tiene la programación dentro de este campo. Se explica de forma general cómo el uso de algoritmos y simulaciones permite estudiar sistemas astrofísicos como cuerpos celestes que no pueden ser reproducidos experimentalmente en la Tierra.}
        \small

        \vspace{0.8em}
    }%
}


\section{Objetivos}
   \begin{itemize}
       \item Identificar algunas áreas de estudio dentro de la astrofísica.
       \item Explicar el papel de la programación en el análisis de datos y simulaciones.
       \item Relacionar la programación con la formación académica en física.
   \end{itemize}

\section{Marco Teórico}
   {La astrofísica es una rama de la física que estudia los cuerpos celestes y los fenómenos que ocurren en el universo, tales como estrellas, galaxias, agujeros negros y la evolución del cosmos. A diferencia de otras áreas de la física, muchos de estos fenómenos no pueden ser observados directamente en laboratorio, por lo que su estudio depende en gran medida del análisis de datos observacionales y de modelos teóricos.}
        \begin{figure}
         
     \begin{center}
        \includegraphics[width=0.5\linewidth]{img/space.jpg} 
         \caption{Todo modelo matemático aplicable a los cuerpos celestes}
          \label{fig:space}
     \end{center}
     \end{figure} 
   

    {En este contexto, la programación se vuelve una herramienta esencial. Mediante el uso de lenguajes de programación es posible procesar grandes cantidades de datos obtenidos por telescopios, simular sistemas astrofísicos y resolver ecuaciones complejas que describen el comportamiento del universo.}
     \vspace{0.8cm}
     \begin{figure}[H]
         
     \begin{center}
        \includegraphics[width=0.5\linewidth]{img/pyton.jpg} 
         \caption{Un ejemplo de un modelo físico hecho en Python}
          \label{fig:python}
     \end{center}
     \end{figure}
     
     
\section{Diseño Experimental}
    {Como resultado de este análisis, se reconoce que la programación es una habilidad indispensable para el estudio de la astrofísica. Tal y como observamos en la Figura \ref{fig:python} su uso permite ampliar las capacidades de análisis, mejorar la comprensión de fenómenos complejos y facilitar el trabajo con grandes volúmenes de datos astronómicos. Además, se evidencia que el aprendizaje temprano de programación fortalece la formación del físico moderno.}

\section{Resultados}
    {La relación entre la astrofísica y la programación demuestra cómo la física ha evolucionado hacia una ciencia cada vez más interdisciplinaria. El uso de herramientas computacionales no solo mejora la precisión de los estudios, sino que también abre nuevas posibilidades de investigación. Por ello, la programación deja de ser un complemento y se convierte en una parte fundamental del trabajo científico.}

\section{Discusión de Resultados}
    {La relación entre la astrofísica y la programación demuestra cómo la física ha evolucionado hacia una ciencia cada vez más interdisciplinaria. El uso de herramientas computacionales no solo mejora la precisión de los estudios, sino que también abre nuevas posibilidades de investigación. Por ello, la programación deja de ser un complemento y se convierte en una parte fundamental del trabajo científico.}
\section{Conclusiones}
    {La astrofísica representa un área de gran interés debido a su capacidad de explicar el origen y la evolución del universo. Para su estudio, la programación resulta esencial, ya que permite analizar datos observacionales y desarrollar modelos teóricos complejos. En conclusión, la programación es una herramienta clave en la formación académica de un físico interesado en la investigación astrofísica.}
\section{Referencias}
     \begin{itemize}
         \item Carroll, B. W., \& Ostlie, D. A. An Introduction to Modern Astrophysics.
         \item Neil deGrassse, Astrophysics for People in A Hurry
         \item Apuntes del curso de Programación 1.
     \end{itemize}
\section{Anexos}
  \subsection{Anexo A}
   {En este anexo se describen algunos de los lenguajes de programación más utilizados en el área de la astrofísica y su aplicación en la investigación científica.}
   \vspace{0.6cm}
   {Python es uno de los lenguajes más empleados debido a su simplicidad y a la gran cantidad de bibliotecas científicas disponibles, como NumPy, SciPy, Matplotlib y Astropy. Estas herramientas permiten analizar datos observacionales, realizar simulaciones numéricas y visualizar resultados de forma eficiente.}
\vspace{0.6cm}
   {Otros lenguajes como C y C++ se utilizan cuando se requiere un alto rendimiento computacional, especialmente en simulaciones de gran escala, como la evolución de galaxias o la dinámica de partículas. Por su parte, lenguajes como Fortran siguen siendo relevantes en códigos científicos clásicos utilizados en astronomía computacional.}
\vspace{0.6cm}
\begin{center}
  \includegraphics[width=0.5\linewidth]{img/Lenguajes.png}   
\end{center}

   \subsection{Anexo B}

    {Este anexo presenta un ejemplo conceptual de una simulación astrofísica. Las simulaciones permiten modelar fenómenos que no pueden reproducirse experimentalmente en la Tierra, como la formación de estrellas, la evolución de galaxias o el movimiento de cuerpos bajo la influencia de la gravedad.}
 \vspace{0.6cm}
    {Mediante el uso de modelos matemáticos y algoritmos computacionales, es posible representar el comportamiento de sistemas astronómicos a lo largo del tiempo. Estos modelos ayudan a los científicos a comprobar teorías, predecir comportamientos futuros y comparar resultados con datos observacionales obtenidos por telescopios.}
 \vspace{0.6cm}
    {La programación juega un papel clave en este proceso, ya que permite implementar los modelos físicos, manejar grandes volúmenes de datos y realizar cálculos complejos de forma eficiente.}
 \vspace{0.6cm}

    \begin{center}
  \includegraphics[width=0.5\linewidth]{img/Simulacion.png}   
    \end{center}

   
\end{document}



